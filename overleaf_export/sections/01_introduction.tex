\section{Introduction}

\subsection{Background}

Medical image analysis is a critical component of modern diagnostics. However, medical images often suffer from noise introduced during acquisition, transmission, or storage. Edge-preserving filters are essential preprocessing tools that reduce noise while maintaining sharp edges and important anatomical details necessary for accurate diagnosis.

\subsection{Edge-Preserving Filters}

Edge-preserving filters form a fundamental category of image processing techniques. Unlike traditional smoothing filters that blur entire regions uniformly, edge-preserving filters selectively smooth while maintaining discontinuities at object boundaries.

\subsubsection{Bilateral Filter}
The bilateral filter \cite{Tomasi1998} is a non-linear filter that combines spatial proximity with intensity similarity. It applies Gaussian weights based on both:
\begin{equation}
BF(x) = \frac{1}{W_p} \sum_{x_i \in \Omega} f(x_i) \cdot G_\sigma(||x - x_i||) \cdot G_\tau(|I(x) - I(x_i)|)
\end{equation}

where $G_\sigma$ represents the spatial Gaussian kernel and $G_\tau$ represents the intensity Gaussian kernel.

\subsubsection{Median Filter}
The median filter replaces each pixel with the median value from its neighborhood. For a kernel of size $k \times k$:
\begin{equation}
MF(x,y) = \text{median}\{f(i,j) : (i,j) \in \text{neighborhood}(x,y)\}
\end{equation}

This filter is particularly effective for salt-and-pepper noise while maintaining sharp transitions at edges.

\subsubsection{Guided Filter}
The guided filter \cite{He2013} uses a guidance image to preserve edges. When using the image itself as guidance:
\begin{equation}
GF(x) = \frac{1}{|W|}\sum_{i \in W(x)} \left(a_k I(i) + b_k\right)
\end{equation}

where coefficients $a_k$ and $b_k$ are computed to minimize local error.

\subsection{Segmentation Methods}

Image segmentation divides an image into homogeneous regions. We evaluated three fundamentally different approaches:

\subsubsection{K-Means Clustering}
K-Means \cite{MacQueen1967} partitions pixels into $k$ clusters by minimizing within-cluster variance:
\begin{equation}
J = \sum_{i=1}^{k} \sum_{x \in C_i} ||x - \mu_i||^2
\end{equation}

where $C_i$ is cluster $i$ and $\mu_i$ is its centroid.

\subsubsection{Split-and-Merge}
This hierarchical method \cite{Horowitz1976} recursively:
1. Splits regions that exceed a uniformity threshold
2. Merges adjacent regions with similar statistics

\subsubsection{Otsu's Thresholding}
Otsu's method \cite{Otsu1979} finds the optimal threshold $t$ that minimizes within-class variance:
\begin{equation}
\sigma_w^2(t) = \omega_1(t)\sigma_1^2(t) + \omega_2(t)\sigma_2^2(t)
\end{equation}

where $\omega_i$ are class weights and $\sigma_i^2$ are class variances.

\subsection{Evaluation Metrics}

\subsubsection{Peak Signal-to-Noise Ratio (PSNR)}
\begin{equation}
\text{PSNR} = 10 \log_{10}\left(\frac{\text{MAX}_I^2}{\text{MSE}}\right)
\end{equation}

where $\text{MAX}_I = 255$ for 8-bit images and MSE is mean squared error.

\subsubsection{Structural Similarity Index (SSIM)}
SSIM \cite{Wang2004} measures perceived quality by considering luminance, contrast, and structure:
\begin{equation}
\text{SSIM}(x,y) = \frac{(2\mu_x\mu_y + c_1)(2\sigma_{xy} + c_2)}{(\mu_x^2 + \mu_y^2 + c_1)(\sigma_x^2 + \sigma_y^2 + c_2)}
\end{equation}
