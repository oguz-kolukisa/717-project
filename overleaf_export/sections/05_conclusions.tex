\section{Conclusions}

\subsection{Summary of Findings}

1. \textbf{Median Filter is Optimal}: Across all test conditions with real medical images, the median filter significantly outperforms bilateral and guided filters, with improvements up to 131\%.

2. \textbf{5×5 Kernel for Detail Preservation}: For clinical diagnosis where preserving anatomical detail is critical, the 5×5 median filter provides the best balance between noise reduction (PSNR 33.3 dB) and detail preservation.

3. \textbf{15×15 Kernel for Robustness}: When dealing with heavy noise (0.8 probability), the 15×15 kernel provides better robustness, though at the cost of some detail loss.

4. \textbf{K-Means for Segmentation}: Combined with median pre-filtering, K-Means clustering provides the most effective segmentation across different image types and noise levels.

5. \textbf{Pre-filtering is Essential}: Segmentation quality dramatically improves with pre-filtering, with unfiltered noisy images producing unusable results.

\subsection{Clinical Implications}

For real medical image processing:

\begin{enumerate}
    \item \textbf{Standard Preprocessing}: Implement 5×5 median filtering as default preprocessing
    \item \textbf{Adaptive Filtering}: Use 15×15 kernel only when noise level is confirmed above 0.5
    \item \textbf{Multi-Method Validation}: Apply multiple segmentation methods and compare results
    \item \textbf{Quality Assurance}: Always validate filtered and segmented images with original data
    \item \textbf{Workflow}: Filter → Segment → Validate → Clinical Interpretation
\end{enumerate}

\subsection{Limitations and Future Work}

\textbf{Limitations}:
\begin{itemize}
    \item Guided filter implementation limited by OpenCV capabilities
    \item Uniform noise model may not reflect all real medical noise patterns
    \item Segmentation evaluation is primarily qualitative
    \item Limited to binary or low-order clustering
\end{itemize}

\textbf{Future Directions}:
\begin{itemize}
    \item Implement advanced filters (anisotropic diffusion, non-local means)
    \item Evaluate on diverse medical imaging modalities (CT, MRI, ultrasound)
    \item Develop quantitative segmentation metrics
    \item Incorporate machine learning for adaptive filter parameter selection
    \item Clinical validation with radiologists
\end{itemize}
