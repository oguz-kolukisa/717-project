\section{Discussion}

\subsection{Key Findings}

\subsubsection{Median Filter Superiority}

The median filter consistently outperforms both bilateral and guided filters:

\begin{equation}
\text{Improvement}_{\text{median}} = \frac{\text{PSNR}_{\text{median}} - \text{PSNR}_{\text{bilateral}}}{\text{PSNR}_{\text{bilateral}}} \times 100\%
\end{equation}

For the medium detail image at 0.1 noise with 5×5 kernel:
\begin{equation}
\text{Improvement} = \frac{40.23 - 17.39}{17.39} \times 100\% = 131.3\%
\end{equation}

This exceptional performance is due to the median filter's ability to preserve edges while removing noise from the neighborhood.

\subsubsection{Real Medical Image Characteristics}

Real medical images exhibit different properties compared to synthetic images:

\begin{enumerate}
    \item \textbf{Resolution Variation}: Resolution ranges from 263×191 to 1433×1534 pixels
    \item \textbf{Complex Anatomy}: High-detail images contain overlapping structures requiring careful processing
    \item \textbf{Natural Noise Patterns}: Real medical noise differs from synthetic uniform noise
    \item \textbf{Diagnostic Importance}: Fine details are critical for clinical interpretation
\end{enumerate}

\subsubsection{Noise Tolerance Analysis}

\begin{table}[H]
\centering
\caption{Average PSNR Across All Images (Median Filter)}
\begin{tabular}{cccc}
\toprule
\textbf{Noise Level} & \textbf{5×5 PSNR} & \textbf{15×15 PSNR} & \textbf{Diagnostic Quality} \\
\midrule
0.1 & 33.3 dB & 25.6 dB & Excellent \\
0.5 & 20.2 dB & 21.4 dB & Moderate \\
0.8 & 11.8 dB & 12.3 dB & Poor \\
\bottomrule
\end{tabular}
\end{table}

\subsection{Segmentation Method Analysis}

\subsubsection{K-Means Clustering}

\textbf{Advantages}:
\begin{itemize}
    \item Fast computation and simple to implement
    \item Effective tissue separation when intensities are well-separated
    \item Flexible number of clusters for multi-tissue analysis
\end{itemize}

\textbf{Disadvantages}:
\begin{itemize}
    \item Sensitive to initialization (random restarts recommended)
    \item Cannot handle variable-shaped clusters
    \item Requires noise reduction for reliable results
\end{itemize}

\textbf{Clinical Application}: Tissue classification in well-controlled medical images (CT, MRI with preprocessing)

\subsubsection{Split-and-Merge}

\textbf{Advantages}:
\begin{itemize}
    \item Hierarchical approach naturally captures multi-scale structures
    \item Adapts to local image characteristics
    \item Good for complex anatomical regions
\end{itemize}

\textbf{Disadvantages}:
\begin{itemize}
    \item More computationally intensive
    \item Block-like artifacts in results
    \item Threshold-dependent uniformity criterion
\end{itemize}

\textbf{Clinical Application}: Hierarchical anatomical structure extraction

\subsubsection{Otsu's Thresholding}

\textbf{Advantages}:
\begin{itemize}
    \item Fully automatic, no parameter tuning
    \item Fast execution
    \item Optimal for bimodal distributions
\end{itemize}

\textbf{Disadvantages}:
\begin{itemize}
    \item Binary segmentation only
    \item Assumes bimodal histogram
    \item Cannot handle gradual intensity transitions
\end{itemize}

\textbf{Clinical Application}: Quick ROI (Region of Interest) detection and foreground/background separation

\subsection{Optimal Configuration Recommendation}

Based on comprehensive analysis:

\begin{table}[H]
\centering
\caption{Recommended Configurations by Use Case}
\begin{tabular}{lll}
\toprule
\textbf{Use Case} & \textbf{Filter Configuration} & \textbf{Segmentation Method} \\
\midrule
Clinical Diagnosis & 5×5 Median & K-Means (2-3 clusters) \\
Research Analysis & 15×15 Median & Split-and-Merge \\
Quick ROI Detection & 5×5 Median & Otsu Thresholding \\
Noisy Data (>0.5) & 15×15 Median & K-Means \\
\bottomrule
\end{tabular}
\end{table}
