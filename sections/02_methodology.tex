\section{Methodology}

\subsection{Experimental Setup}

\subsubsection{Test Images}
Three real medical images were extracted from the provided RAR archive:

\begin{table}[H]
\centering
\caption{Test Images Specifications}
\begin{tabular}{lccl}
\toprule
\textbf{Image} & \textbf{File} & \textbf{Resolution} & \textbf{Description} \\
\midrule
Low Detail & m\_low.jpg & 263 × 191 & Simpler medical scan \\
Medium Detail & m\_mid.jpg & 1433 × 1534 & Standard resolution image \\
High Detail & m\_high.jpg & 948 × 902 & Complex anatomical structures \\
\bottomrule
\end{tabular}
\end{table}

\subsubsection{Noise Model}
Uniform noise was applied to simulate realistic sensor noise. The probability parameter indicates the fraction of pixels affected:

\begin{itemize}
    \item \textbf{0.1 probability}: 10\% of pixels randomly set to values in [0, 255]
    \item \textbf{0.5 probability}: 50\% of pixels affected (significant corruption)
    \item \textbf{0.8 probability}: 80\% of pixels affected (severe corruption)
\end{itemize}

\subsubsection{Step 1: Filter Evaluation}
Configuration: 3 images × 3 noise levels × 2 kernel sizes × 3 filters = 54 test cases

\textbf{Filters evaluated}:
\begin{itemize}
    \item \textit{Bilateral Filter} \cite{Tomasi1998}: Edge-preserving smoothing using spatial and intensity distances
    \item \textit{Median Filter}: Non-linear filtering preserving edges through rank-order operations
    \item \textit{Guided Filter} \cite{He2013}: Structure-preserving filter using guidance images
\end{itemize}

Kernel sizes tested: 5×5 and 15×15

Evaluation: PSNR and SSIM \cite{Wang2004} computed against original clean images

\subsubsection{Step 2: Segmentation Evaluation}
Configuration: 3 images × 4 noise conditions × 3 filter options × 3 methods = 108 test cases

Noise conditions: None, 0.1, 0.5, 0.8

Filter options: None, 5×5 median, 15×15 median

\textbf{Segmentation methods evaluated}:
\begin{itemize}
    \item \textit{K-Means Clustering} \cite{MacQueen1967}: Partitioning pixels into $k$ classes based on intensity similarity
    \item \textit{Otsu's Thresholding} \cite{Otsu1979}: Automatic binary threshold selection using histogram analysis
    \item \textit{Split-and-Merge Algorithm} \cite{Horowitz1976}: Hierarchical region-based segmentation with quadtree decomposition
\end{itemize}

\subsection{Implementation}

The project was implemented in Python 3.11 using:
\begin{itemize}
    \item OpenCV for filtering and basic image operations
    \item scikit-learn for K-Means clustering
    \item scikit-image for advanced image processing
    \item scipy for scientific computations
\end{itemize}

Key implementation details:

\begin{lstlisting}
def median_filter_ep(image, kernel_size=5):
    return cv2.medianBlur(image.astype(np.uint8), kernel_size)

def bilateral_filter(image, d=9, sigma_color=75, sigma_space=75):
    return cv2.bilateralFilter(image.astype(np.uint8), 
                              d, sigma_color, sigma_space)

def kmeans_segmentation(image, n_clusters=2):
    pixels = image.reshape(-1, 1).astype(np.float32)
    criteria = (cv2.TERM_CRITERIA_EPS + 
                cv2.TERM_CRITERIA_MAX_ITER, 100, 0.2)
    _, labels, centers = cv2.kmeans(pixels, n_clusters, None, 
                                    criteria, 10, 
                                    cv2.KMEANS_RANDOM_CENTERS)
    centers = np.uint8(centers)
    segmented = centers[labels.flatten()]
    return segmented.reshape(image.shape)
\end{lstlisting}
