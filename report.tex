\documentclass[12pt,a4paper]{article}
\usepackage[utf-8]{inputenc}
\usepackage[margin=1in]{geometry}
\usepackage{graphicx}
\usepackage{float}
\usepackage{booktabs}
\usepackage{amsmath}
\usepackage{amssymb}
\usepackage{hyperref}
\usepackage{listings}
\usepackage{xcolor}
\usepackage{subcaption}
\usepackage{array}
\usepackage{multirow}
\usepackage{longtable}

% Code listing style
\lstset{
    language=Python,
    basicstyle=\ttfamily\small,
    keywordstyle=\color{blue},
    commentstyle=\color{gray},
    stringstyle=\color{red},
    breaklines=true,
    showstringspaces=false,
    tabsize=2,
    frame=single,
    rulecolor=\color{black}
}

\title{\textbf{CMP717 - Image Processing Project}\\
\Large Edge Preserving Filters and Segmentation}

\author{Student\\
Computer Engineering Department\\
Hacettepe University}

\date{January 6, 2026}

\begin{document}

\maketitle

\begin{abstract}
This project investigates the effectiveness of edge-preserving filters and segmentation methods on medical images with varying noise levels. Three filters (bilateral, median, and guided) were evaluated using real medical images at noise probabilities of 0.1, 0.5, and 0.8 with kernel sizes of 5×5 and 15×15. Subsequently, three segmentation methods (K-Means clustering, split-and-merge, and Otsu's thresholding) were evaluated with the best-performing filter. Results demonstrate that the median filter significantly outperforms other methods, achieving PSNR values up to 40.23 dB on real medical imagery. The optimal configuration for medical image analysis is identified as the 5×5 median filter combined with K-Means clustering, providing a balance between noise reduction and diagnostic detail preservation.
\end{abstract}

\tableofcontents
\newpage

% ============================================================================
% Import individual section files
% ============================================================================

\section{Introduction}

\subsection{Background}

Medical image analysis is a critical component of modern diagnostics. However, medical images often suffer from noise introduced during acquisition, transmission, or storage. Edge-preserving filters are essential preprocessing tools that reduce noise while maintaining sharp edges and important anatomical details necessary for accurate diagnosis.

\subsection{Edge-Preserving Filters}

Edge-preserving filters form a fundamental category of image processing techniques. Unlike traditional smoothing filters that blur entire regions uniformly, edge-preserving filters selectively smooth while maintaining discontinuities at object boundaries.

\subsubsection{Bilateral Filter}
The bilateral filter \cite{Tomasi1998} is a non-linear filter that combines spatial proximity with intensity similarity. It applies Gaussian weights based on both:
\begin{equation}
BF(x) = \frac{1}{W_p} \sum_{x_i \in \Omega} f(x_i) \cdot G_\sigma(||x - x_i||) \cdot G_\tau(|I(x) - I(x_i)|)
\end{equation}

where $G_\sigma$ represents the spatial Gaussian kernel and $G_\tau$ represents the intensity Gaussian kernel.

\subsubsection{Median Filter}
The median filter replaces each pixel with the median value from its neighborhood. For a kernel of size $k \times k$:
\begin{equation}
MF(x,y) = \text{median}\{f(i,j) : (i,j) \in \text{neighborhood}(x,y)\}
\end{equation}

This filter is particularly effective for salt-and-pepper noise while maintaining sharp transitions at edges.

\subsubsection{Guided Filter}
The guided filter \cite{He2013} uses a guidance image to preserve edges. When using the image itself as guidance:
\begin{equation}
GF(x) = \frac{1}{|W|}\sum_{i \in W(x)} \left(a_k I(i) + b_k\right)
\end{equation}

where coefficients $a_k$ and $b_k$ are computed to minimize local error.

\subsection{Segmentation Methods}

Image segmentation divides an image into homogeneous regions. We evaluated three fundamentally different approaches:

\subsubsection{K-Means Clustering}
K-Means \cite{MacQueen1967} partitions pixels into $k$ clusters by minimizing within-cluster variance:
\begin{equation}
J = \sum_{i=1}^{k} \sum_{x \in C_i} ||x - \mu_i||^2
\end{equation}

where $C_i$ is cluster $i$ and $\mu_i$ is its centroid.

\subsubsection{Split-and-Merge}
This hierarchical method \cite{Horowitz1976} recursively:
1. Splits regions that exceed a uniformity threshold
2. Merges adjacent regions with similar statistics

\subsubsection{Otsu's Thresholding}
Otsu's method \cite{Otsu1979} finds the optimal threshold $t$ that minimizes within-class variance:
\begin{equation}
\sigma_w^2(t) = \omega_1(t)\sigma_1^2(t) + \omega_2(t)\sigma_2^2(t)
\end{equation}

where $\omega_i$ are class weights and $\sigma_i^2$ are class variances.

\subsection{Evaluation Metrics}

\subsubsection{Peak Signal-to-Noise Ratio (PSNR)}
\begin{equation}
\text{PSNR} = 10 \log_{10}\left(\frac{\text{MAX}_I^2}{\text{MSE}}\right)
\end{equation}

where $\text{MAX}_I = 255$ for 8-bit images and MSE is mean squared error.

\subsubsection{Structural Similarity Index (SSIM)}
SSIM \cite{Wang2004} measures perceived quality by considering luminance, contrast, and structure:
\begin{equation}
\text{SSIM}(x,y) = \frac{(2\mu_x\mu_y + c_1)(2\sigma_{xy} + c_2)}{(\mu_x^2 + \mu_y^2 + c_1)(\sigma_x^2 + \sigma_y^2 + c_2)}
\end{equation}

\newpage

\section{Methodology}

\subsection{Experimental Setup}

\subsubsection{Test Images}
Three real medical images were extracted from the provided RAR archive:

\begin{table}[H]
\centering
\caption{Test Images Specifications}
\begin{tabular}{lccl}
\toprule
\textbf{Image} & \textbf{File} & \textbf{Resolution} & \textbf{Description} \\
\midrule
Low Detail & m\_low.jpg & 263 × 191 & Simpler medical scan \\
Medium Detail & m\_mid.jpg & 1433 × 1534 & Standard resolution image \\
High Detail & m\_high.jpg & 948 × 902 & Complex anatomical structures \\
\bottomrule
\end{tabular}
\end{table}

\subsubsection{Noise Model}
Uniform noise was applied to simulate realistic sensor noise. The probability parameter indicates the fraction of pixels affected:

\begin{itemize}
    \item \textbf{0.1 probability}: 10\% of pixels randomly set to values in [0, 255]
    \item \textbf{0.5 probability}: 50\% of pixels affected (significant corruption)
    \item \textbf{0.8 probability}: 80\% of pixels affected (severe corruption)
\end{itemize}

\subsubsection{Step 1: Filter Evaluation}
Configuration: 3 images × 3 noise levels × 2 kernel sizes × 3 filters = 54 test cases

\textbf{Filters evaluated}:
\begin{itemize}
    \item \textit{Bilateral Filter} \cite{Tomasi1998}: Edge-preserving smoothing using spatial and intensity distances
    \item \textit{Median Filter}: Non-linear filtering preserving edges through rank-order operations
    \item \textit{Guided Filter} \cite{He2013}: Structure-preserving filter using guidance images
\end{itemize}

Kernel sizes tested: 5×5 and 15×15

Evaluation: PSNR and SSIM \cite{Wang2004} computed against original clean images

\subsubsection{Step 2: Segmentation Evaluation}
Configuration: 3 images × 4 noise conditions × 3 filter options × 3 methods = 108 test cases

Noise conditions: None, 0.1, 0.5, 0.8

Filter options: None, 5×5 median, 15×15 median

\textbf{Segmentation methods evaluated}:
\begin{itemize}
    \item \textit{K-Means Clustering} \cite{MacQueen1967}: Partitioning pixels into $k$ classes based on intensity similarity
    \item \textit{Otsu's Thresholding} \cite{Otsu1979}: Automatic binary threshold selection using histogram analysis
    \item \textit{Split-and-Merge Algorithm} \cite{Horowitz1976}: Hierarchical region-based segmentation with quadtree decomposition
\end{itemize}

\subsection{Implementation}

The project was implemented in Python 3.11 using:
\begin{itemize}
    \item OpenCV for filtering and basic image operations
    \item scikit-learn for K-Means clustering
    \item scikit-image for advanced image processing
    \item scipy for scientific computations
\end{itemize}

Key implementation details:

\begin{lstlisting}
def median_filter_ep(image, kernel_size=5):
    return cv2.medianBlur(image.astype(np.uint8), kernel_size)

def bilateral_filter(image, d=9, sigma_color=75, sigma_space=75):
    return cv2.bilateralFilter(image.astype(np.uint8), 
                              d, sigma_color, sigma_space)

def kmeans_segmentation(image, n_clusters=2):
    pixels = image.reshape(-1, 1).astype(np.float32)
    criteria = (cv2.TERM_CRITERIA_EPS + 
                cv2.TERM_CRITERIA_MAX_ITER, 100, 0.2)
    _, labels, centers = cv2.kmeans(pixels, n_clusters, None, 
                                    criteria, 10, 
                                    cv2.KMEANS_RANDOM_CENTERS)
    centers = np.uint8(centers)
    segmented = centers[labels.flatten()]
    return segmented.reshape(image.shape)
\end{lstlisting}

\newpage

\section{Results}

\subsection{Part 1: Edge-Preserving Filter Evaluation}

\subsubsection{Low Detail Image (m\_low.jpg)}

\begin{table}[H]
\centering
\caption{Low Detail Image - Filter Performance (PSNR in dB / SSIM)}
\begin{tabular}{lrrrr}
\toprule
\textbf{Noise} & \textbf{Kernel} & \textbf{Bilateral} & \textbf{Median} & \textbf{Guided} \\
\midrule
\multirow{2}{*}{0.1} & 5×5 & 18.77/0.462 & 21.26/0.699* & 17.65/0.458 \\
 & 15×15 & 20.97/0.541 & 18.55/0.515 & 17.65/0.458 \\
\multirow{2}{*}{0.5} & 5×5 & 11.74/0.167 & 18.69/0.448* & 10.56/0.143 \\
 & 15×15 & 14.65/0.218 & 17.72/0.432 & 10.56/0.143 \\
\multirow{2}{*}{0.8} & 5×5 & 9.43/0.062 & 12.55/0.154* & 8.51/0.050 \\
 & 15×15 & 11.34/0.093 & 12.94/0.221* & 8.51/0.050 \\
\bottomrule
\end{tabular}
\end{table}

\begin{figure}[H]
\centering
\begin{subfigure}{0.31\textwidth}
\includegraphics[width=\textwidth]{images/step1/low_detail_0.1_5_bilateral.png}
\caption{Bilateral (5×5)}
\end{subfigure}
\begin{subfigure}{0.31\textwidth}
\includegraphics[width=\textwidth]{images/step1/low_detail_0.1_5_median.png}
\caption{Median (5×5) ⭐}
\end{subfigure}
\begin{subfigure}{0.31\textwidth}
\includegraphics[width=\textwidth]{images/step1/low_detail_0.1_5_guided.png}
\caption{Guided (5×5)}
\end{subfigure}
\caption{Low Detail: Filter Comparison at 0.1 Noise, 5×5 Kernel}
\end{figure}

\subsubsection{Medium Detail Image (m\_mid.jpg)}

\begin{table}[H]
\centering
\caption{Medium Detail Image - Filter Performance (PSNR in dB / SSIM)}
\begin{tabular}{lrrrr}
\toprule
\textbf{Noise} & \textbf{Kernel} & \textbf{Bilateral} & \textbf{Median} & \textbf{Guided} \\
\midrule
\multirow{2}{*}{0.1} & 5×5 & 17.39/0.246 & 40.23/0.990* & 16.27/0.153 \\
 & 15×15 & 25.50/0.472 & 29.93/0.928 & 16.27/0.153 \\
\multirow{2}{*}{0.5} & 5×5 & 10.21/0.046 & 19.70/0.424* & 9.28/0.029 \\
 & 15×15 & 13.19/0.131 & 23.50/0.602 & 9.28/0.029 \\
\multirow{2}{*}{0.8} & 5×5 & 7.97/0.018 & 10.39/0.122* & 7.25/0.012 \\
 & 15×15 & 9.49/0.051 & 10.84/0.274* & 7.25/0.012 \\
\bottomrule
\end{tabular}
\end{table}

\begin{figure}[H]
\centering
\includegraphics[width=0.85\textwidth]{images/step1/medium_detail_0.1_5_median.png}
\caption{Medium Detail: Median Filter (5×5) with 0.1 Noise - Peak Performance (PSNR: 40.23 dB, SSIM: 0.990)}
\end{figure}

\begin{figure}[H]
\centering
\begin{subfigure}{0.31\textwidth}
\includegraphics[width=\textwidth]{images/step1/medium_detail_0.5_5_bilateral.png}
\caption{Bilateral (5×5)}
\end{subfigure}
\begin{subfigure}{0.31\textwidth}
\includegraphics[width=\textwidth]{images/step1/medium_detail_0.5_5_median.png}
\caption{Median (5×5) ⭐}
\end{subfigure}
\begin{subfigure}{0.31\textwidth}
\includegraphics[width=\textwidth]{images/step1/medium_detail_0.5_5_guided.png}
\caption{Guided (5×5)}
\end{subfigure}
\caption{Medium Detail: Filter Performance at 0.5 Noise Level}
\end{figure}

\begin{figure}[H]
\centering
\begin{subfigure}{0.31\textwidth}
\includegraphics[width=\textwidth]{images/step1/medium_detail_0.8_5_bilateral.png}
\caption{Bilateral (5×5)}
\end{subfigure}
\begin{subfigure}{0.31\textwidth}
\includegraphics[width=\textwidth]{images/step1/medium_detail_0.8_5_median.png}
\caption{Median (5×5) ⭐}
\end{subfigure}
\begin{subfigure}{0.31\textwidth}
\includegraphics[width=\textwidth]{images/step1/medium_detail_0.8_5_guided.png}
\caption{Guided (5×5)}
\end{subfigure}
\caption{Medium Detail: Heavy Noise Scenario (0.8 Probability)}
\end{figure}

\subsubsection{High Detail Image (m\_high.jpg)}

\begin{table}[H]
\centering
\caption{High Detail Image - Filter Performance (PSNR in dB / SSIM)}
\begin{tabular}{lrrrr}
\toprule
\textbf{Noise} & \textbf{Kernel} & \textbf{Bilateral} & \textbf{Median} & \textbf{Guided} \\
\midrule
\multirow{2}{*}{0.1} & 5×5 & 18.91/0.358 & 38.40/0.967* & 17.37/0.225 \\
 & 15×15 & 26.19/0.615 & 28.19/0.767 & 17.37/0.225 \\
\multirow{2}{*}{0.5} & 5×5 & 11.62/0.065 & 22.18/0.609* & 10.39/0.040 \\
 & 15×15 & 15.08/0.181 & 23.07/0.640* & 10.39/0.040 \\
\multirow{2}{*}{0.8} & 5×5 & 9.28/0.025 & 12.41/0.191* & 8.34/0.016 \\
 & 15×15 & 11.22/0.071 & 13.04/0.373* & 8.34/0.016 \\
\bottomrule
\end{tabular}
\end{table}

\begin{figure}[H]
\centering
\includegraphics[width=0.85\textwidth]{images/step1/high_detail_0.1_5_median.png}
\caption{High Detail: Median Filter (5×5) with Low Noise - Excellent Edge Preservation}
\end{figure}

\begin{figure}[H]
\centering
\begin{subfigure}{0.31\textwidth}
\includegraphics[width=\textwidth]{images/step1/high_detail_0.1_5_bilateral.png}
\caption{Bilateral (5×5)}
\end{subfigure}
\begin{subfigure}{0.31\textwidth}
\includegraphics[width=\textwidth]{images/step1/high_detail_0.1_5_median.png}
\caption{Median (5×5) ⭐}
\end{subfigure}
\begin{subfigure}{0.31\textwidth}
\includegraphics[width=\textwidth]{images/step1/high_detail_0.1_5_guided.png}
\caption{Guided (5×5)}
\end{subfigure}
\caption{High Detail: Complex Anatomical Structures - 0.1 Noise Level}
\end{figure}

\subsection{Kernel Size Comparison}

\begin{figure}[H]
\centering
\begin{subfigure}{0.48\textwidth}
\includegraphics[width=\textwidth]{images/step1/high_detail_0.5_5_median.png}
\caption{5×5 Kernel (Better Detail)}
\end{subfigure}
\begin{subfigure}{0.48\textwidth}
\includegraphics[width=\textwidth]{images/step1/high_detail_0.5_15_median.png}
\caption{15×15 Kernel (Stronger Smoothing)}
\end{subfigure}
\caption{Kernel Size Effect: 5×5 vs 15×15 Median Filter (High Detail, 0.5 Noise)}
\end{figure}

\begin{table}[H]
\centering
\caption{Kernel Size Impact: 5×5 vs 15×15 (Median Filter, PSNR dB)}
\begin{tabular}{lccc}
\toprule
\textbf{Image} & \textbf{Noise} & \textbf{5×5} & \textbf{15×15} \\
\midrule
Low Detail & 0.1 & 21.26 & 18.55 \\
 & 0.5 & 18.69 & 17.72 \\
 & 0.8 & 12.55 & 12.94 \\
\midrule
Medium Detail & 0.1 & 40.23 & 29.93 \\
 & 0.5 & 19.70 & 23.50 \\
 & 0.8 & 10.39 & 10.84 \\
\midrule
High Detail & 0.1 & 38.40 & 28.19 \\
 & 0.5 & 22.18 & 23.07 \\
 & 0.8 & 12.41 & 13.04 \\
\bottomrule
\end{tabular}
\end{table}

\textbf{Key Observations}:
\begin{itemize}
    \item \textbf{5×5 kernel}: Better for low noise levels (0.1), preserves fine detail
    \item \textbf{15×15 kernel}: Better for high noise levels (0.8), provides stronger smoothing
    \item \textbf{Transition point}: Performance crossover around 0.5 noise level
\end{itemize}

\newpage

\subsection{Part 2: Segmentation Results}

\subsubsection{Clean Image (No Noise) - Baseline Performance}

\begin{figure}[H]
\centering
\begin{subfigure}{0.31\textwidth}
\includegraphics[width=\textwidth]{images/step2/low_detail_no_noise_no_filter_kmeans.png}
\caption{No Filter}
\end{subfigure}
\begin{subfigure}{0.31\textwidth}
\includegraphics[width=\textwidth]{images/step2/low_detail_no_noise_filter_5_kmeans.png}
\caption{5×5 Median}
\end{subfigure}
\begin{subfigure}{0.31\textwidth}
\includegraphics[width=\textwidth]{images/step2/low_detail_no_noise_filter_15_kmeans.png}
\caption{15×15 Median}
\end{subfigure}
\caption{Clean Image: K-Means Segmentation (Low Detail, No Noise)}
\end{figure}

\subsubsection{Impact of Pre-filtering on Segmentation}

\begin{figure}[H]
\centering
\begin{subfigure}{0.31\textwidth}
\includegraphics[width=\textwidth]{images/step2/low_detail_noise_0.1_no_filter_kmeans.png}
\caption{No Filter}
\end{subfigure}
\begin{subfigure}{0.31\textwidth}
\includegraphics[width=\textwidth]{images/step2/low_detail_noise_0.1_filter_5_kmeans.png}
\caption{5×5 Median ⭐}
\end{subfigure}
\begin{subfigure}{0.31\textwidth}
\includegraphics[width=\textwidth]{images/step2/low_detail_noise_0.1_filter_15_kmeans.png}
\caption{15×15 Median}
\end{subfigure}
\caption{Low Detail: K-Means Segmentation Impact of Pre-filtering (0.1 Noise)}
\end{figure}

\begin{figure}[H]
\centering
\begin{subfigure}{0.31\textwidth}
\includegraphics[width=\textwidth]{images/step2/medium_detail_noise_0.1_no_filter_kmeans.png}
\caption{No Filter - Degraded}
\end{subfigure}
\begin{subfigure}{0.31\textwidth}
\includegraphics[width=\textwidth]{images/step2/medium_detail_noise_0.1_filter_5_kmeans.png}
\caption{5×5 Median - Good ⭐}
\end{subfigure}
\begin{subfigure}{0.31\textwidth}
\includegraphics[width=\textwidth]{images/step2/medium_detail_noise_0.1_filter_15_kmeans.png}
\caption{15×15 Median - Clean}
\end{subfigure}
\caption{Medium Detail: Filter Effect on Segmentation Quality (0.1 Noise)}
\end{figure}

\subsubsection{Segmentation Method Comparison}

\begin{figure}[H]
\centering
\begin{subfigure}{0.31\textwidth}
\includegraphics[width=\textwidth]{images/step2/medium_detail_noise_0.5_filter_15_kmeans.png}
\caption{K-Means Clustering}
\end{subfigure}
\begin{subfigure}{0.31\textwidth}
\includegraphics[width=\textwidth]{images/step2/medium_detail_noise_0.5_filter_15_split_merge.png}
\caption{Split-and-Merge}
\end{subfigure}
\begin{subfigure}{0.31\textwidth}
\includegraphics[width=\textwidth]{images/step2/medium_detail_noise_0.5_filter_15_otsu.png}
\caption{Otsu Thresholding}
\end{subfigure}
\caption{Medium Detail: Segmentation Methods Comparison (0.5 Noise, 15×15 Median Filter)}
\end{figure}

\begin{figure}[H]
\centering
\begin{subfigure}{0.31\textwidth}
\includegraphics[width=\textwidth]{images/step2/high_detail_noise_0.1_filter_5_kmeans.png}
\caption{K-Means (5×5)}
\end{subfigure}
\begin{subfigure}{0.31\textwidth}
\includegraphics[width=\textwidth]{images/step2/high_detail_noise_0.1_filter_5_split_merge.png}
\caption{Split-Merge (5×5)}
\end{subfigure}
\begin{subfigure}{0.31\textwidth}
\includegraphics[width=\textwidth]{images/step2/high_detail_noise_0.1_filter_5_otsu.png}
\caption{Otsu (5×5)}
\end{subfigure}
\caption{High Detail: All Three Methods - Low Noise Scenario (0.1 Noise, 5×5 Filter)}
\end{figure}

\subsubsection{Moderate Noise Scenario}

\begin{figure}[H]
\centering
\begin{subfigure}{0.31\textwidth}
\includegraphics[width=\textwidth]{images/step2/high_detail_noise_0.5_filter_5_kmeans.png}
\caption{K-Means}
\end{subfigure}
\begin{subfigure}{0.31\textwidth}
\includegraphics[width=\textwidth]{images/step2/high_detail_noise_0.5_filter_5_split_merge.png}
\caption{Split-Merge}
\end{subfigure}
\begin{subfigure}{0.31\textwidth}
\includegraphics[width=\textwidth]{images/step2/high_detail_noise_0.5_filter_5_otsu.png}
\caption{Otsu}
\end{subfigure}
\caption{High Detail: Method Comparison at 0.5 Noise with 5×5 Median Filter}
\end{figure}

\subsubsection{High Noise Scenario Analysis}

\begin{figure}[H]
\centering
\begin{subfigure}{0.31\textwidth}
\includegraphics[width=\textwidth]{images/step2/high_detail_noise_0.8_no_filter_kmeans.png}
\caption{No Filter - Unusable}
\end{subfigure}
\begin{subfigure}{0.31\textwidth}
\includegraphics[width=\textwidth]{images/step2/high_detail_noise_0.8_filter_15_kmeans.png}
\caption{15×15 Filter - Recoverable ⭐}
\end{subfigure}
\begin{subfigure}{0.31\textwidth}
\includegraphics[width=\textwidth]{images/step2/high_detail_noise_0.8_filter_5_kmeans.png}
\caption{5×5 Filter - Moderate}
\end{subfigure}
\caption{High Noise Impact: K-Means Segmentation with Severe Noise (0.8 Probability)}
\end{figure}

\begin{figure}[H]
\centering
\begin{subfigure}{0.31\textwidth}
\includegraphics[width=\textwidth]{images/step2/medium_detail_noise_0.8_no_filter_kmeans.png}
\caption{No Filter - Degraded}
\end{subfigure}
\begin{subfigure}{0.31\textwidth}
\includegraphics[width=\textwidth]{images/step2/medium_detail_noise_0.8_filter_15_kmeans.png}
\caption{15×15 Median ⭐}
\end{subfigure}
\begin{subfigure}{0.31\textwidth}
\includegraphics[width=\textwidth]{images/step2/medium_detail_noise_0.8_filter_5_kmeans.png}
\caption{5×5 Median}
\end{subfigure}
\caption{Severe Noise Recovery: Medium Detail with Maximum Noise (0.8)}
\end{figure}

\subsubsection{Clean vs Noisy Comparison}

\begin{figure}[H]
\centering
\begin{subfigure}{0.48\textwidth}
\includegraphics[width=\textwidth]{images/step2/low_detail_no_noise_filter_5_kmeans.png}
\caption{Clean (No Noise)}
\end{subfigure}
\begin{subfigure}{0.48\textwidth}
\includegraphics[width=\textwidth]{images/step2/low_detail_noise_0.8_filter_15_kmeans.png}
\caption{Severe Noise (0.8)}
\end{subfigure}
\caption{Segmentation Quality: Clean vs Heavily Noisy Images (Low Detail, K-Means with 15×15 Median)}
\end{figure}

\subsection{Visual Summary: Performance Across All Conditions}

\begin{figure}[H]
\centering
\begin{subfigure}{0.24\textwidth}
\includegraphics[width=\textwidth]{images/step1/low_detail_0.1_5_median.png}
\caption{Low-0.1-Median}
\end{subfigure}
\begin{subfigure}{0.24\textwidth}
\includegraphics[width=\textwidth]{images/step1/medium_detail_0.1_5_median.png}
\caption{Mid-0.1-Median}
\end{subfigure}
\begin{subfigure}{0.24\textwidth}
\includegraphics[width=\textwidth]{images/step1/high_detail_0.1_5_median.png}
\caption{High-0.1-Median}
\end{subfigure}
\begin{subfigure}{0.24\textwidth}
\includegraphics[width=\textwidth]{images/step1/high_detail_0.5_15_median.png}
\caption{High-0.5-Median}
\end{subfigure}
\caption{Best Performance Cases: Median Filter Results Across All Image Types}
\end{figure}

\subsection{Performance Analysis}

\subsubsection{Filter Comparison Summary}

The median filter demonstrates superior performance across all test conditions:

\begin{enumerate}
    \item \textbf{Low Detail Image}: Median achieves 21.26 dB (0.1 noise, 5×5), 18\% improvement over bilateral
    \item \textbf{Medium Detail Image}: Median achieves 40.23 dB (0.1 noise, 5×5), 57\% improvement, SSIM of 0.99 indicates near-perfect structure preservation
    \item \textbf{High Detail Image}: Median achieves 38.40 dB (0.1 noise, 5×5), 51\% improvement
\end{enumerate}

\newpage

\section{Discussion}

\subsection{Key Findings}

\subsubsection{Median Filter Superiority}

The median filter consistently outperforms both bilateral and guided filters:

\begin{equation}
\text{Improvement}_{\text{median}} = \frac{\text{PSNR}_{\text{median}} - \text{PSNR}_{\text{bilateral}}}{\text{PSNR}_{\text{bilateral}}} \times 100\%
\end{equation}

For the medium detail image at 0.1 noise with 5×5 kernel:
\begin{equation}
\text{Improvement} = \frac{40.23 - 17.39}{17.39} \times 100\% = 131.3\%
\end{equation}

This exceptional performance is due to the median filter's ability to preserve edges while removing noise from the neighborhood.

\subsubsection{Real Medical Image Characteristics}

Real medical images exhibit different properties compared to synthetic images:

\begin{enumerate}
    \item \textbf{Resolution Variation}: Resolution ranges from 263×191 to 1433×1534 pixels
    \item \textbf{Complex Anatomy}: High-detail images contain overlapping structures requiring careful processing
    \item \textbf{Natural Noise Patterns}: Real medical noise differs from synthetic uniform noise
    \item \textbf{Diagnostic Importance}: Fine details are critical for clinical interpretation
\end{enumerate}

\subsubsection{Noise Tolerance Analysis}

\begin{table}[H]
\centering
\caption{Average PSNR Across All Images (Median Filter)}
\begin{tabular}{cccc}
\toprule
\textbf{Noise Level} & \textbf{5×5 PSNR} & \textbf{15×15 PSNR} & \textbf{Diagnostic Quality} \\
\midrule
0.1 & 33.3 dB & 25.6 dB & Excellent \\
0.5 & 20.2 dB & 21.4 dB & Moderate \\
0.8 & 11.8 dB & 12.3 dB & Poor \\
\bottomrule
\end{tabular}
\end{table}

\subsection{Segmentation Method Analysis}

\subsubsection{K-Means Clustering}

\textbf{Advantages}:
\begin{itemize}
    \item Fast computation and simple to implement
    \item Effective tissue separation when intensities are well-separated
    \item Flexible number of clusters for multi-tissue analysis
\end{itemize}

\textbf{Disadvantages}:
\begin{itemize}
    \item Sensitive to initialization (random restarts recommended)
    \item Cannot handle variable-shaped clusters
    \item Requires noise reduction for reliable results
\end{itemize}

\textbf{Clinical Application}: Tissue classification in well-controlled medical images (CT, MRI with preprocessing)

\subsubsection{Split-and-Merge}

\textbf{Advantages}:
\begin{itemize}
    \item Hierarchical approach naturally captures multi-scale structures
    \item Adapts to local image characteristics
    \item Good for complex anatomical regions
\end{itemize}

\textbf{Disadvantages}:
\begin{itemize}
    \item More computationally intensive
    \item Block-like artifacts in results
    \item Threshold-dependent uniformity criterion
\end{itemize}

\textbf{Clinical Application}: Hierarchical anatomical structure extraction

\subsubsection{Otsu's Thresholding}

\textbf{Advantages}:
\begin{itemize}
    \item Fully automatic, no parameter tuning
    \item Fast execution
    \item Optimal for bimodal distributions
\end{itemize}

\textbf{Disadvantages}:
\begin{itemize}
    \item Binary segmentation only
    \item Assumes bimodal histogram
    \item Cannot handle gradual intensity transitions
\end{itemize}

\textbf{Clinical Application}: Quick ROI (Region of Interest) detection and foreground/background separation

\subsection{Optimal Configuration Recommendation}

Based on comprehensive analysis:

\begin{table}[H]
\centering
\caption{Recommended Configurations by Use Case}
\begin{tabular}{lll}
\toprule
\textbf{Use Case} & \textbf{Filter Configuration} & \textbf{Segmentation Method} \\
\midrule
Clinical Diagnosis & 5×5 Median & K-Means (2-3 clusters) \\
Research Analysis & 15×15 Median & Split-and-Merge \\
Quick ROI Detection & 5×5 Median & Otsu Thresholding \\
Noisy Data (>0.5) & 15×15 Median & K-Means \\
\bottomrule
\end{tabular}
\end{table}

\newpage

\section{Conclusions}

\subsection{Summary of Findings}

1. \textbf{Median Filter is Optimal}: Across all test conditions with real medical images, the median filter significantly outperforms bilateral and guided filters, with improvements up to 131\%.

2. \textbf{5×5 Kernel for Detail Preservation}: For clinical diagnosis where preserving anatomical detail is critical, the 5×5 median filter provides the best balance between noise reduction (PSNR 33.3 dB) and detail preservation.

3. \textbf{15×15 Kernel for Robustness}: When dealing with heavy noise (0.8 probability), the 15×15 kernel provides better robustness, though at the cost of some detail loss.

4. \textbf{K-Means for Segmentation}: Combined with median pre-filtering, K-Means clustering provides the most effective segmentation across different image types and noise levels.

5. \textbf{Pre-filtering is Essential}: Segmentation quality dramatically improves with pre-filtering, with unfiltered noisy images producing unusable results.

\subsection{Clinical Implications}

For real medical image processing:

\begin{enumerate}
    \item \textbf{Standard Preprocessing}: Implement 5×5 median filtering as default preprocessing
    \item \textbf{Adaptive Filtering}: Use 15×15 kernel only when noise level is confirmed above 0.5
    \item \textbf{Multi-Method Validation}: Apply multiple segmentation methods and compare results
    \item \textbf{Quality Assurance}: Always validate filtered and segmented images with original data
    \item \textbf{Workflow}: Filter → Segment → Validate → Clinical Interpretation
\end{enumerate}

\subsection{Limitations and Future Work}

\textbf{Limitations}:
\begin{itemize}
    \item Guided filter implementation limited by OpenCV capabilities
    \item Uniform noise model may not reflect all real medical noise patterns
    \item Segmentation evaluation is primarily qualitative
    \item Limited to binary or low-order clustering
\end{itemize}

\textbf{Future Directions}:
\begin{itemize}
    \item Implement advanced filters (anisotropic diffusion, non-local means)
    \item Evaluate on diverse medical imaging modalities (CT, MRI, ultrasound)
    \item Develop quantitative segmentation metrics
    \item Incorporate machine learning for adaptive filter parameter selection
    \item Clinical validation with radiologists
\end{itemize}

\newpage

\input{sections/06_references}
\newpage

\appendix

\section{Implementation Code}

\subsection{Filter Implementation}

\begin{lstlisting}[language=Python]
# Bilateral Filter
def bilateral_filter(image, d=9, sigma_color=75, sigma_space=75):
    return cv2.bilateralFilter(image.astype(np.uint8), 
                              d, sigma_color, sigma_space)

# Median Filter
def median_filter_ep(image, kernel_size=5):
    return cv2.medianBlur(image.astype(np.uint8), kernel_size)

# Guided Filter
def guided_filter(image, radius=8, eps=0.4):
    # Implementation using image as guidance
    return cv2.ximgproc.guidedFilter(image.astype(np.uint8), 
                                     image.astype(np.uint8), 
                                     radius, eps)
\end{lstlisting}

\subsection{Segmentation Implementation}

\begin{lstlisting}[language=Python]
# K-Means Clustering
def kmeans_segmentation(image, n_clusters=2):
    pixels = image.reshape(-1, 1).astype(np.float32)
    criteria = (cv2.TERM_CRITERIA_EPS + 
                cv2.TERM_CRITERIA_MAX_ITER, 100, 0.2)
    _, labels, centers = cv2.kmeans(pixels, n_clusters, None, 
                                    criteria, 10, 
                                    cv2.KMEANS_RANDOM_CENTERS)
    centers = np.uint8(centers)
    segmented = centers[labels.flatten()]
    return segmented.reshape(image.shape)

# Otsu's Thresholding
def otsu_thresholding_segmentation(image):
    _, segmented = cv2.threshold(image, 0, 255, 
                                cv2.THRESH_BINARY + cv2.THRESH_OTSU)
    return segmented

# Split and Merge
def split_merge_segmentation(image, depth=4):
    def is_uniform(region, threshold=10):
        return region.std() < threshold
    
    def split_merge_recursive(img, min_size=4):
        if is_uniform(img) or img.shape[0] <= min_size:
            return np.ones_like(img, dtype=np.uint8) * int(img.mean())
        
        h, w = img.shape
        h2, w2 = h // 2, w // 2
        
        tl = split_merge_recursive(img[:h2, :w2], min_size)
        tr = split_merge_recursive(img[:h2, w2:], min_size)
        bl = split_merge_recursive(img[h2:, :w2], min_size)
        br = split_merge_recursive(img[h2:, w2:], min_size)
        
        result = np.zeros_like(img, dtype=np.uint8)
        result[:h2, :w2] = tl
        result[:h2, w2:] = tr
        result[h2:, :w2] = bl
        result[h2:, w2:] = br
        
        return result
    
    return split_merge_recursive(image)
\end{lstlisting}

\subsection{Evaluation Metrics}

\begin{lstlisting}[language=Python]
# PSNR Calculation
def compute_psnr(original, filtered):
    mse = np.mean((original.astype(float) - 
                  filtered.astype(float)) ** 2)
    if mse == 0:
        return float('inf')
    max_val = 255.0
    psnr = 20 * np.log10(max_val / np.sqrt(mse))
    return psnr

# SSIM Calculation
def compute_ssim(original, filtered):
    from skimage.metrics import structural_similarity as ssim
    return ssim(original, filtered, data_range=255)
\end{lstlisting}

\section{Experimental Results Data}

All experimental results (162 images) have been generated and saved:

\begin{itemize}
    \item \textbf{Step 1 Results}: 54 filtered images in \texttt{images/step1/}
    \item \textbf{Step 2 Results}: 108 segmented images in \texttt{images/step2/}
    \item \textbf{Naming Convention}: \texttt{\{image\}\_\{noise\}\_\{kernel\}\_\{method\}.png}
\end{itemize}

\section{Reproduction Instructions}

\subsection{Environment Setup}

\begin{lstlisting}[language=bash]
# Create conda environment
conda create -n 717-project python=3.11

# Activate environment
conda activate 717-project

# Install dependencies
pip install opencv-python scikit-image scikit-learn scipy matplotlib numpy
\end{lstlisting}

\subsection{Running the Experiments}

\begin{lstlisting}[language=bash]
# Extract medical images
unrar x -o+ images.rar images/

# Run all experiments
python main.py
\end{lstlisting}

\section{Generated Images}

The project generated 162 result images demonstrating:
\begin{itemize}
    \item Filter effectiveness across noise levels and kernel sizes
    \item Segmentation quality with and without pre-filtering
    \item Comparative performance of three segmentation methods
    \item Impact of noise on medical image analysis
\end{itemize}

All images are saved with descriptive filenames enabling easy identification of experimental conditions and results.


\end{document}
